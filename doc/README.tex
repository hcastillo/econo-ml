\documentclass{article}
\usepackage[utf8]{inputenc}
\usepackage[english]{babel}
\usepackage{hyperref}
\usepackage{listings}
\usepackage{xcolor}
\usepackage{amssymb}
\usepackage{graphicx}


\title{econo-ml: Reinforce Learning Policy Recommendation for Interbank Network Stability}
\author{}
\date{}

\begin{document}

\maketitle

\section{Auxiliary files}
\begin{itemize}
    \item \texttt{requirements.txt}: list of the necessary python packages
\end{itemize}

\section{Interbank model}

\begin{itemize}
    \item \texttt{interbank.py}: use to execute standalone the Interbank simulation.
    \begin{itemize}
        \item It accepts command line options. For instance:
            \begin{lstlisting}[language=bash, basicstyle=\ttfamily\small]
interbank.py --log DEBUG --n 150 --t 2000
interbank.py --save results.gdt --p 0.5 eta=0.35 param=X
            \end{lstlisting}
        \item When it is used as a package, the sequence should be:
            \begin{lstlisting}[language=Python, basicstyle=\ttfamily\small]
import interbank
model = interbank.Model()
model.config.configure(param=x)
model.forward()
eta = model.get_current_fitness()
model.set_policy_recommendation(eta=0.5)
            \end{lstlisting}
    \end{itemize}
    \item Basic options:
        \begin{lstlisting}[language=bash, basicstyle=\ttfamily\small]
        # To list all options:
        interbank.py --help

        # Using lender's change mechanism ShockedMarket3
        # with probability of attachment 0.3:
        interbank.py --lc ShockedMarket3 --p 0.3

        # Same for Preferential with m nodes:
        interbank.py --lc Preferential --m 0.3

        # To use a fastest algorithm to run in big simulations:
        interbank.py --fast

        # To run a simulation based on exp_runner:
        python -m experiments.exp_shockedmarket --do
        \end{lstlisting}

    \item \texttt{colab\_interbank.ipynb}: Notebook version of the standalone \texttt{interbank.py} with the same results but plotted using Bokeh.
    \item \texttt{interbank\_lenderchange.py}: It contains the different algorithms that control the change of lender in the model.
    \item \texttt{exp\_runner.py}: A prototype for executing experiments with different parameters and using MonteCarlo (using concurrent.futures to allow multiple threads).
    \item \texttt{exp\_runner\_distributed.py}: A sub-prototype that uses ray library to execute in a cluster.
    \item \texttt{exp\_runner\_no\_concurrent.py}: Another sub-prototype that avoids the use of parallelism.
    \item \texttt{exp\_runner\_no\_concurrent.py}: Another sub-prototype that avoids the use of parallelism.
    \item \texttt{exp\_runner\_comparer.py}: A derivation of the former prototype though to compare the evolution with \texttt{p} (probability of attachment in an Erdos-Renyi graph) in the \texttt{x} axis and other parameters accross the \texttt{y} axis.
    \item \texttt{exp\_runner\_surviving.py}: A derivation of the former prototype using ray library to execute in a cluster.
    \item \texttt{experiments/}: directory with all the experiments conducted. The results of that executions are stored in a folder determined inside each experiment.
    \item \texttt{utils/plot\_psi.py}: Generate a table of axis\_x x axis\_y plots.
    \item \texttt{utils/labplot2\_interbank.lml}: \href{https://labplot.org/}{LabPlot2} file to plot the results of the \texttt{interbank.py}. By the way the best way is to use \href{https://gretl.sourceforge.net/}{Gretl} as an export format.
    \item \texttt{algorithm.drawio} and \texttt{algorithm.drawio.pf}: the \href{https://www.drawio.com/}{draw.io} and PDF schema of the algorithm used in the model to propagate shocks and to balance sheets.
\end{itemize}

\section{RL with Stable Baselines3}
\begin{itemize}
    \item \texttt{interbank\_agent.py}: agent to test using PPO
    \item \texttt{run\_ppo.py}: run and simulate with PPO agent
    \item \texttt{run\_td3.py}: run and simulate with TD3 algorithm
    \item \texttt{models/XXXX.zip}: instances of Gymnasium.env trained to use with \texttt{run\_XXXX.py}
    \item \texttt{utils/plot\_ppo.py}: auxiliary creator of plots to play the results of PPO
    \item Usage:
\begin{lstlisting}[language=bash, basicstyle=\ttfamily\small]
# train first and save the model env:
run_ppo.py --train ppo_10000 --t 10000 --verbose

# use the trained env and generate a simulation of T=1000
# with Interbank model
run_ppo.py --load ppo_10000 --save results_ppo.txt
\end{lstlisting}
\end{itemize}

\section{Basic usage of the model}


\begin{figure}[htb]
  \centering
  \includegraphics[width=\textwidth,keepaspectratio]{algorithm}
  \caption{Sequence of steps: grey boxes indicates moments in which that statistic is obtained}
  \label{fig:algorithm}
\end{figure}


\begin{itemize}
    \item \texttt{interbank.py --seed 1234 --t 500 --p 0.2}: Execute the model with $T=500$ and $LenderChange$ algorithm of $ShockedMarket3$ with an Erdös-Réni with probability of attachment of $0.2$ and using a seed for generating random values of $1234$ (same results if you generate again with other equal parameters and repeat this integer number for seed).
    \item \texttt{interbank.py --save file.gdt --log DEBUG --logfile file.txt}: Save the results in \texttt{file.gdt} in $CSV$ and the detailed log in \texttt{file.txt}.
    \item \texttt{interbank.py --save file.gdt --stats\_market --detail\_banks 5,7}: Save the results in \texttt{file.gdt}, a second file \texttt{fileb.gdt} with the results for only banks and times participating really in the loans market is generated, and also a third file \texttt{file\_detailed.gdt} with the concrete statistics for banks 5 and 7. With \texttt{--detail\_times 10,12} all specific details for all banks in times 10 and 12 are present in this third file.
\end{itemize}

\section{Statistics}

Different statistics can be obtained after running the model, either in \textbf{csv} output, or in \textbf{gdt} (Gretl format).
This statistics collect data in each time for the average or individually, depending on the usage. Possible statistics obtained from the model are:

\begin{itemize}
    \item \textbf{active\_borrowers}: Number of banks that are involved in a loan as borrowers. Both values in global and \textbf{stats\_market} will be the same.
    \item \textbf{active\_lenders}: Number of banks that are involved in a loan as borrowers. Both values in global and \textbf{stats\_market} will be the same.
    \item \textbf{asset\_i}: Assets of the lender of this bank ($D + E$)
    \item \textbf{asset\_j}: Assets of the borrowers of this bank ($D + E$)
    \item \textbf{bad\_debt}: Sum of the bad debt
    \item \textbf{bankruptcies}: Number of banks that failed in this step
    \item \textbf{bankrupcty\_rationed}: Number of banks that failed in this step due to rationing
    \item \textbf{best\_lender}: ID of the bank which more connections in the graph
    \item \textbf{best\_lender\_clients}: Number of banks connected with the best lender
    \item \textbf{c}: Lender capacity ($1 - \frac{E}{maxE}$) of the bank
    \item \textbf{communities}: Subsets of nodes with higher internal edge density than connections to the rest of the graph
    \item \textbf{communities\_not\_alone}:  Number of \textbf{communities} that are not formed by only one node
    \item \textbf{deposits}: Deposits $D$ in the balance $L + C + R = D + E$
    \item \textbf{equity}: Equity $E$ of the bank: $L + C + R = D + E$
    \item \textbf{fitness}: Fitness ($\mu$) of the bank
    \item \textbf{gcs}: When we use an Erdös–Rényi graph, the Giant Component Size is the largest number of nodes that are interconnected.
    \item \textbf{grade\_avg}: Average number of edges (connections) for the total banks
    \item \textbf{incrementD}: Amount of ($ \Delta D$) for the bank
    \item \textbf{interest\_rate}: Interest rate $r$ of the bank
    \item \textbf{l\_equity}: Log of equity ($log(E)$)
    \item \textbf{leverage}: Financial leverage ($l/E$) of the bank considering only the banks that are inside a loan, named \textbf{leverage\_} in Gretl due to name restrictions of the environment.
    \item \textbf{liquidity}: Total liquidity $L$ of the Banks  $L + C + R = D + E$
    \item \textbf{loans}: Amount borrowed by the bank
    \item \textbf{num\_banks}: Number of banks currently surviving in the model (interesting when  \textbf{allow\_replacement\_of\_bankrupted=False})
    \item \textbf{num\_loans}:  Num of loans in this step. Both global and \textbf{stats values \_market} will be the same
    \item \textbf{num\_of\_rationed}: Number of banks that were rationed in this step (needed money and were without any possible lender)
    \item \textbf{policy}: Policy recommendation $\eta$ of the system in the range $[0..1]$. As $\eta$ is a global value, the same number applies for all banks.
    \item \textbf{potential\_credit\_channels}:  Considering there is a graph of connections between banks, then \textbf{number\_of\_edges()} in the graph
    \item \textbf{potential\_lenders}: Number of banks in the first shock having a possitive shock ($ \Delta D$)
    \item \textbf{prob\_bankruptcy}: Probability of bankruptcy $p=\frac{E}{E_{max}}$, between $[0..1]$
    \item \textbf{profits}: Profits obtained in that step
    \item \textbf{psi}: Power market ($psi$) value $[0..1]$
    \item \textbf{rationing}: Total amounf of the loans $l$ of the banks
    \item \textbf{real\_t}: Times in which are no loans are removed in the extra statistics generated when we use \textbf{--stats\_market}. Real $t$ instants of time are stored in this variable to track when were really those values are obtained in the original statistics.
    \item \textbf{reserves}: Reserves $R$ in the balance $L + C + R = D + E$
    \item \textbf{systemic\_leverage}: Financial leverage but considering in the mean the total banks of the model $N$
\end{itemize}

\begin{table}[h]
\centering
\begin{tabular}{|l|c|c|c|c|c}
\hline
Name & Type & Global & \textbf{stats\_market} & Individual & Graphs \\
\textbf{active\_borrowers} & integer & $\checkmark$ &  $\checkmark$  &  & \\
\textbf{active\_lenders}   & integer & $\checkmark$ & $\checkmark$  &  & \\
\textbf{asset\_i}          & float & $\overline{x}/0 $ & $\overline{x}/0$ & $\checkmark$ &  \\
\textbf{asset\_j}          & float & $\overline{x}/0 $ & $\overline{x}/0$ & $\checkmark$ &  \\
\textbf{bad\_debt}         & float & $ \sum$ &  $ \sum$ & $\checkmark$ & \\
\textbf{bankruptcies}      & integer & $ \sum$ &  $ \sum$ & $\checkmark$ &  \\
\textbf{bankrupcty\_rationed} & integer & $ \sum$ &  $ \sum$ &  &  \\
\textbf{best\_lender}      &  integer &  $\checkmark$ & $\checkmark$  &  & \\
\textbf{best\_lender\_clients}  & integer &  $\checkmark$ &  $\checkmark$ &  & \\
\textbf{c}                 & float &  $\overline{x}/nan $ & $\overline{x}/nan $ &  &  \\
\textbf{communities} & integer &  &   &  & $\checkmark$\\
\textbf{communities\_not\_alone} & integer &  &   &  & $\checkmark$ \\
\textbf{deposits}          & float & $ \sum$ &  $ \sum$ & $\checkmark$ &  \\
\textbf{equity}            & float &  $ \sum$ &  $ \sum$ &  \\
\textbf{eta}            & float &  $ $\checkmark$ & $\checkmark$  &  \\
\textbf{fitness}           & float & $\overline{x}$ & $\overline{x}/nan $ &  $\checkmark$  & \\
\textbf{gcs}               & integer &  &   &  & $\checkmark$ \\
\textbf{grade\_avg}        & integer &  &   &  & $\checkmark$\\
\textbf{incrementD}        &  float &   $\sum$  &  $\sum$  &  &  $\checkmark$  \\
\textbf{interest\_rate}    &  float &  $\overline{x}/0$ & $\overline{x}/nan $  & $\checkmark & $ \\
\textbf{l\_equity}         & float &  $ \sum$ &  $ \sum$ &  \\
\textbf{leverage} / \textbf{leverage\_}         & float  &  $\overline{x}$ & $\overline{x}/nan $  & $\checkmark$  &  \\
\textbf{liquidity}         &  float &  $\sum$ &  $\sum$ &  $\checkmark$ &    \\
\textbf{loans}             & float &  $ \sum$ &  $ \sum$ & $\checkmark$ &  \\
\textbf{num\_banks}        &  integer &  $\checkmark$ & $\checkmark$ &  & \\
\textbf{num\_loans}        &  integer &  $ \checkmark$ &  $\checkmark$ & $\checkmark$ &  \\
\textbf{num\_of\_rationed} & integer &  $\checkmark$ & $\checkmark$ &  & \\
\textbf{policy}            & float &  $\checkmark$ & $\checkmark$ &  & \\
\textbf{potential\_credit\_channels} & integer &  $\checkmark$ & $\checkmark$ &  & \\
\textbf{potential\_lenders} & integer &  $\checkmark$ & $\checkmark$ &  & \\
\textbf{prob\_bankruptcy}  &  float &  $\checkmark$ & $\checkmark$  &  $\checkmark$  & \\
\textbf{profits}           & float &  $ \sum$ &  $ \sum$ &  & $\checkmark$ \\
\textbf{psi}               &  float &  $\checkmark/0 $ & $\checkmark/nan $ &  & $\checkmark$ \\
\textbf{rationing}         &  float &  $\sum$ &  $\sum$ &  & $\checkmark$  \\
\textbf{real\_t}         &  integer &  &  $\checkmark$ &  &   \\
\textbf{reserves}          & float & $ \sum$ &  $ \sum$ &  & $\checkmark$ \\
\textbf{systemic\_leverage} & float &  $\overline{x}$ & $ \overline{x} $  &  &  \\
\hline
\end{tabular}
\caption{$\checkmark$=value without any modification. $\sum$=sum of the value for all banks. $ \overline{x}$ = average of the value for all banks. $0$ = No banks in this statistic. $nan$=Instead of zero, the value of "not a number" is used}
\end{table}

\begin{itemize}
        \item Global: using \texttt{--save}: each data column marked in the table with "Global" column will be obtained for all the $N$ banks in the model for all instants time $T$ (rows)
        \item With \textbf{--stats\_market} what we obtain will be statistics for the subsets of banks that in each time are engaged in a real loan. So if in the time $t$ there are no loans, it is removed from this statistics. The special value \texttt{real\_t} indicates which was the original time.
        \item Individual is data obtained when we use \textbf{--detail\_times} or \textbf{--detail\_banks} and it stores statistics of those moments for all the banks individually or specific banks.
        \item Graphs are data obtained when we have a \textbf{LenderChange} algorithm with a random graph, in which we can determine for each time it is generated specific data.
    \end{itemize}
\end{document}
