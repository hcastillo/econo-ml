\documentclass{article}
\usepackage[utf8]{inputenc}
\usepackage[english]{babel}
\usepackage{hyperref}
\usepackage{listings}
\usepackage{xcolor}
\usepackage{amssymb}
\usepackage{graphicx}


\title{econo-ml: Reinforce Learning Policy Recommendation for Interbank Network Stability}
\author{}
\date{}

\begin{document}

\maketitle

\section{Auxiliary files}
\begin{itemize}
    \item \texttt{requirements.txt}: list of the necessary python packages
\end{itemize}

\section{Interbank model}

\begin{itemize}
    \item \texttt{interbank.py}: use to execute standalone the Interbank simulation.
    \begin{itemize}
        \item It accepts command line options. For instance:
\begin{lstlisting}[language=bash, basicstyle=\ttfamily\small]
interbank.py --log DEBUG --n 150 --t 2000
interbank.py --save results.gdt --p 0.5 eta=0.35 param=X
\end{lstlisting}
        \item When it is used as a package, the sequence should be:
\begin{lstlisting}[language=Python, basicstyle=\ttfamily\small]
import interbank
model = interbank.Model()
model.config.configure( param=x )
model.forward()
μ = model.get_current_fitness()
model.set_policy_recommendation( ŋ=0.5 )
\end{lstlisting}
    \end{itemize}

    \item Basic options:
\begin{lstlisting}[language=bash, basicstyle=\ttfamily\small]
# To list all options:
interbank.py --help

# Using lender's change mechanism ShockedMarket3 with probability of attachment 0.3:
interbank.py --lc ShockedMarket3 --p 0.3

# Same for Preferential with m nodes:
interbank.py --lc Preferential --m 0.3

# To use a fastest algorithm to run in big simulations:
interbank.py --fast

# To run a simulation based on exp_runner:
python -m experiments.exp_shockedmarket --do
\end{lstlisting}

    \item \texttt{colab\_interbank.ipynb}: Notebook version of the standalone \texttt{interbank.py} with the same results but plotted using Bokeh.
    \item \texttt{interbank\_lenderchange.py}: It contains the different algorithms that control the change of lender in the model.
    \item \texttt{exp\_runner.py}: A prototype for executing experiments with different parameters and using MonteCarlo (using concurrent.futures to allow multiple threads).
    \item \texttt{exp\_runner\_distributed.py}: A sub-prototype that uses ray library to execute in a cluster.
    \item \texttt{exp\_runner\_no\_concurrent.py}: Another sub-prototype that avoids the use of parallelism.
    \item \texttt{exp\_runner\_no\_concurrent.py}: Another sub-prototype that avoids the use of parallelism.
    \item \texttt{exp\_runner\_comparer.py}: A derivation of the former prototype though to compare the evolution with \texttt{p} (probability of attachment in an Erdos-Renyi graph) in the \texttt{x} axis and other parameters accross the \texttt{y} axis.
    \item \texttt{exp\_runner\_surviving.py}: A derivation of the former prototype using ray library to execute in a cluster.
    \item \texttt{experiments/}: directory with all the experiments conducted. The results of that executions are stored in a folder determined inside each experiment.
    \item \texttt{utils/plot\_psi.py}: Generate a table of axis\_x x axis\_y plots.
    \item \texttt{utils/labplot2\_interbank.lml}: \href{https://labplot.org/}{LabPlot2} file to plot the results of the \texttt{interbank.py}. By the way the best way is to use \href{https://gretl.sourceforge.net/}{Gretl} as an export format.
    \item \texttt{algorithm.drawio} and \texttt{algorithm.drawio.pf}: the \href{https://www.drawio.com/}{draw.io} and PDF schema of the algorithm used in the model to propagate shocks and to balance sheets.
\end{itemize}

\section{RL with Stable Baselines3}
\begin{itemize}
    \item \texttt{interbank\_agent.py}: agent to test using PPO
    \item \texttt{run\_ppo.py}: run and simulate with PPO agent
    \item \texttt{run\_td3.py}: run and simulate with TD3 algorithm
    \item \texttt{models/XXXX.zip}: instances of Gymnasium.env trained to use with \texttt{run\_XXXX.py}
    \item \texttt{utils/plot\_ppo.py}: auxiliary creator of plots to play the results of PPO
    \item Usage:
\begin{lstlisting}[language=bash, basicstyle=\ttfamily\small]
# train first and save the model env:
run_ppo.py --train ppo_10000 --t 10000 --verbose

# use the trained env and generate a simulation of T=1000 with Interbank model
run_ppo.py --load ppo_10000 --save results_ppo.txt
\end{lstlisting}
\end{itemize}

\section{Basic usage of the model}



\begin{figure}[htb]
  \centering
  \includegraphics[width=\textwidth,keepaspectratio]{algorithm}
  \caption{Flujo del modelo interbancario: shocks, préstamos, repagos y ventas forzosas}
  \label{fig:algorithm}
\end{figure}


\begin{itemize}
    \item \texttt{interbank\_agent.py}: agent to test using PPO
    \item Usage:
\end{itemize}


\begin{table}[h]
\centering
\begin{tabular}{|l|c|r|}
\hline
Name & Explanation & Global & \textbf{stats\_market} & Individual & Graphs \\
\hline
\textbf{gcs} & bla bla &  &  &  & $\checkmark$ \\
\hline
\textbf{gcs} & bla bla &  &  &  & $\checkmark$ \\
\hline
\end{tabular}
\caption{Simple table example.}
\end{table}



\end{document}
